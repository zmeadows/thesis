% ******************************************************************************
% ****************************** Custom Margin *********************************

% Add `custommargin' in the document class options to use this section
% Set {innerside margin / outerside margin / topmargin / bottom margin}  and
% other page dimensions
\ifsetCustomMargin
  \RequirePackage[left=37mm,right=30mm,top=35mm,bottom=30mm]{geometry}
  \setFancyHdr % To apply fancy header after geometry package is loaded
\fi

% Add spaces between paragraphs
%\setlength{\parskip}{0.5em}
% Ragged bottom avoids extra whitespaces between paragraphs
\raggedbottom
% To remove the excess top spacing for enumeration, list and description
%\usepackage{enumitem}
%\setlist[enumerate,itemize,description]{topsep=0em}

% *****************************************************************************
% ******************* Fonts (like different typewriter fonts etc.)*************
\usepackage{rotating}
\usepackage{pdflscape}

\usepackage{booktabs,array,ragged2e}
\newcommand{\vname}[1]{\mathrm{#1}} % or: \mathit
\def\sectionautorefname{Section}
\def\subsectionautorefname{Section}
\def\subsubsectionautorefname{Section}
% At best in the preamble:
\DeclareRobustCommand\mytikzdot{\tikz \fill[black] (1ex,1ex) circle (0.5ex);}
\DeclareRobustCommand\mytikzreddot{\tikz \fill[red] (1ex,1ex) circle (0.5ex);}
\DeclareRobustCommand\mytikzbluedot{\tikz \fill[blue] (1ex,1ex) circle (0.5ex);}
% Add `customfont' in the document class option to use this section
%\usepackage{mathtools}
\ifsetCustomFont
%\usepackage{mathtools}
%
%\usepackage[T1]{fontenc}
\usepackage{lmodern}
\newcommand\hmmax{0}
\newcommand\bmmax{0}
%\usepackage[T1]{fontenc}
%\usepackage[notextcomp]{kpfonts}%  for math    
%\usepackage{libertine}%  serif and sans serif
%\usepackage[scaled=0.85]{beramono}%% mono
\usepackage[T1]{fontenc}
\usepackage{newpxtext}% FOR PALATINO TEXT
\usepackage{newpxmath}% FOR MATH
%\RequirePackage{mathpazo} % Palatino with real small caps and old style figures
%\PassOptionsToPackage[osf,sc]{mathpazo}%
%\usepackage[osf,sc]{mathpazo}%
%\usepackage{newpxtext}
\usepackage{newpxmath}
\usepackage{notoccite}% PREVENTS CITES IN CAPTIONS FROM MISNUMBERING YOUR REFERENCES 

%\linespread{1.25} % a bit more for Palatino
\usepackage{bm}
\usepackage{setspace}

\usepackage{enumerate}% http://ctan.org/pkg/enumerate

\usepackage{braket}
\usepackage{makecell}
\usepackage{xspace}

\newcommand*{\vecc}[1]{%
	\if#1\relax\bm{#1}\else\mathbf{#1}\fi
}

%\linespread{1.05} % a bit more for Palatino
  % Set your custom font here and use `customfont' in options. Leave empty to
  % load computer modern font (default LaTeX font).
  %\RequirePackage{helvet}

  % For use with XeLaTeX
  %  \setmainfont[
  %    Path              = ./libertine/opentype/,
  %    Extension         = .otf,
  %    UprightFont = LinLibertine_R,
  %    BoldFont = LinLibertine_RZ, % Linux Libertine O Regular Semibold
  %    ItalicFont = LinLibertine_RI,
  %    BoldItalicFont = LinLibertine_RZI, % Linux Libertine O Regular Semibold Italic
  %  ]
  %  {libertine}
  %  % load font from system font
  %  \newfontfamily\libertinesystemfont{Linux Libertine O}
\fi


%



% *****************************************************************************
% **************************** Custom Packages ********************************

% ************************* Algorithms and Pseudocode **************************

%\usepackage{algpseudocode}


% ********************Captions and Hyperreferencing / URL **********************

% Captions: This makes captions of figures use a boldfaced small font.
%\RequirePackage[small,bf]{caption}

\RequirePackage[labelsep=space,tableposition=top]{caption}
\renewcommand{\figurename}{Fig.} %to support older versions of captions.sty


% *************************** Graphics and figures *****************************

%\usepackage{rotating}
%\usepackage{wrapfig}

% Uncomment the following two lines to force Latex to place the figure.
% Use [H] when including graphics. Note 'H' instead of 'h'
%\usepackage{float}
%\restylefloat{figure}

% Subcaption package is also available in the sty folder you can use that by
% uncommenting the following line
% This is for people stuck with older versions of texlive
%\usepackage{sty/caption/subcaption}
%\usepackage{subcaption}

% ********************************** Tables ************************************
\usepackage{booktabs} % For professional looking tables
\usepackage{multirow}
%\usepackage{multicol}
%\usepackage{longtable}
%\usepackage{tabularx}

\usepackage{breqn}
% *********************************** SI Units *********************************
\usepackage{siunitx} % use this package module for SI units


% ******************************* Line Spacing *********************************

% Choose linespacing as appropriate. Default is one-half line spacing as per the
% University guidelines

% \doublespacing
\onehalfspacing
% \singlespacing


% ************************ Formatting / Footnote *******************************

% Don't break enumeration (etc.) across pages in an ugly manner (default 10000)
%\clubpenalty=500
%\widowpenalty=500

%\usepackage[perpage]{footmisc} %Range of footnote options
\usepackage{subfig}

% *****************************************************************************
% *************************** Bibliography  and References ********************

%\usepackage{cleveref} %Referencing without need to explicitly state fig /table

% Add `custombib' in the document class option to use this section
%\ifuseCustomBib
%   \RequirePackage[square, sort, numbers, authoryear]{natbib} % CustomBib

% If you would like to use biblatex for your reference management, as opposed to the default `natbibpackage` pass the option `custombib` in the document class. Comment out the previous line to make sure you don't load the natbib package. Uncomment the following lines and specify the location of references.bib file

%\RequirePackage[backend=biber, style=numeric-comp, citestyle=numeric, sorting=nty, natbib=true]{biblatex}
%\addbibresource{References/references} %Location of references.bib only for biblatex, Do not omit the .bib extension from the filename.

%\fi


%\usepackage[sectionbib, super, sort]{natbib}
%\usepackage{chapterbib}

% changes the default name `Bibliography` -> `References'
\renewcommand{\bibname}{References}


% ******************************************************************************
% ************************* User Defined Commands ******************************
% ******************************************************************************

% *********** To change the name of Table of Contents / LOF and LOT ************

%\renewcommand{\contentsname}{My Table of Contents}
%\renewcommand{\listfigurename}{My List of Figures}
%\renewcommand{\listtablename}{My List of Tables}


% ********************** TOC depth and numbering depth *************************

\setcounter{secnumdepth}{2}
\setcounter{tocdepth}{2}






% ******************************* Nomenclature *********************************

% To change the name of the Nomenclature section, uncomment the following line

\renewcommand{\nomname}{Glossary}


% ********************************* Appendix ***********************************

% The default value of both \appendixtocname and \appendixpagename is `Appendices'. These names can all be changed via:

%\renewcommand{\appendixtocname}{List of appendices}
%\renewcommand{\appendixname}{Appndx}

% *********************** Configure Draft Mode **********************************

% Uncomment to disable figures in `draft'
%\setkeys{Gin}{draft=true}  % set draft to false to enable figures in `draft'

% These options are active only during the draft mode
% Default text is "Draft"
%\SetDraftText{DRAFT}

% Default Watermark location is top. Location (top/bottom)
%\SetDraftWMPosition{bottom}

% Draft Version - default is v1.0
\SetDraftVersion{v0.1}

% Draft Text grayscale value (should be between 0-black and 1-white)
% Default value is 0.75
%\SetDraftGrayScale{0.8}


%% math enviroments
\let\openbox\relax
\usepackage{amsthm}
\usepackage{amsmath} 
\usepackage{booktabs} 
\usepackage{array} 

\usepackage{color, colortbl}
\usepackage{amsmath}
\usepackage{upgreek}
% quotes
\usepackage{epigraph}
% quotes
\usepackage{url}
\usepackage{doi}

\usepackage{bbold}
%tikz
\usepackage{tikz}
\usepackage{tikz,pgfplots}
\usepackage{pgfplots}
\usetikzlibrary{calc,patterns,decorations.pathmorphing,decorations.markings}
\usetikzlibrary{decorations.markings}
\usetikzlibrary{quotes,angles,positioning}
\usetikzlibrary{arrows}
\usetikzlibrary{tikzmark,calc}
\usetikzlibrary{arrows.meta}
\usetikzlibrary{patterns,matrix,shapes,arrows,fit,positioning,shadows,trees}



\usepackage[listings,theorems,skins,breakable]{tcolorbox}
\tcbuselibrary{xparse,skins,breakable}



\usepackage{dcolumn}
\newcolumntype{d}[1]{D{.}{.}{#1}}  % define "d" column type
\usepackage{siunitx}
\sisetup{table-format=-1.3, table-space-text-post={***}}



% Mathematics
%\DeclareMathAlphabet{\mathbfsf}{\encodingdefault}{\sfdefault}{bx}{sl}

%\newcommand*{\matrix}[1]{\mathbf{#1}}
%\newcommand*{\vecc}[1]{\mathbf{#1}}
\newcommand*{\tensor}[1]{\mathbf{#1}}
%\newcommand{\tensor}[1]{\mathbfsf{#1}}




% Mathematics
\theoremstyle{definition}
\newtheorem*{definition}{Definition}
\newtheorem{thm}{Theorem}
\usepackage{thmtools}
% chapter 7 packages
\usepackage{algorithm,algorithmic}

\usetikzlibrary{pgfplots.patchplots}
\usepgfplotslibrary{patchplots}
\renewcommand*{\figureautorefname}{Fig.}
\numberwithin{equation}{section}

\usepackage{lipsum}
%\usepackage{mathtools}
\usepackage{cuted}
\usepackage{tabularx}% http://ctan.org/pkg/tabularx
\usepackage{hyperref}
\usepackage[all]{hypcap}
\usepackage{xcolor}
\hypersetup{
	colorlinks,
	linkcolor={red!50!black},
	citecolor={blue!50!black},
	urlcolor={blue!80!black}
}
\newcommand{\algorithmautorefname}{Algorithm}

\newtheorem{remark}{Remark}
\newtheorem{theorem}{Theorem}
\newtcolorbox{mybox}[1]{colback=black!5!white,colframe=blue!25!black,fonttitle=\bfseries,title=#1}
\newcolumntype{Y}{>{\raggedleft\arraybackslash}X}% raggedleft column X
% ******************************** Todo Notes **********************************
%% Uncomment the following lines to have todonotes.

\ifsetDraft
	\usepackage[colorinlistoftodos]{todonotes}
	\newcommand{\znote}[1]{\todo[author=zac,size=\small,inline,color=green!40]{#1}}
	\newcommand{\snote}[1]{\todo[author=stephane,size=\small,inline,color=red!40]{#1}}
\else
	\newcommand{\znote}[1]{}
	\newcommand{\snote}[1]{}
	\newcommand{\listoftodos}{}
\fi

% Example todo: \mynote{Hey! I have a note}

% *****************************************************************************
% ******************* Better enumeration my MB*************
\usepackage{enumitem}

% ADDED BY ME (ZAC MEADOWS)

\usepackage{thesis_macros}
